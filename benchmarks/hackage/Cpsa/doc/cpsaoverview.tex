\documentclass[12pt]{article}
\usepackage{alltt}

\newcommand{\cpsacopying}{\begingroup
  \renewcommand{\thefootnote}{}\footnotetext{{\copyright} 2009 The
    MITRE Corporation.  Permission to copy without fee all or part of
    this material is granted provided that the copies are not made or
    distributed for direct commercial advantage, this copyright notice
    and the title of the publication and its date appear, and notice
    in given that copying is by permission of The MITRE
    Corporation.}\endgroup}

\newcommand{\nterm}[1]{\ensuremath{\langle\mathit{#1}\rangle}}
\newcommand{\nterms}[1]{\ensuremath{\nterm{#1}^\ast}}
\newcommand{\ntermp}[1]{\ensuremath{\nterm{#1}^+}}
\newcommand{\ntermo}[1]{#1$^?$}

\title{CPSA Overview}
\author{John D.~Ramsdell\qquad Joshua D.~Guttman\\ The MITRE Corporation}

\begin{document}
\maketitle
\cpsacopying

Enclosed is a brief overview of CPSA, along with a description of
CPSA's support for the rely-guarantee method.  The message terms
\nterm{term} used by CPSA are a straightforward representation of
terms using Lisp-style, prefix notation.

A subset of the terms are called {\em atoms}. Atoms belong to the {\em
  base sorts} \texttt{name, text, data, skey, akey}.  Syntactically,
atomic terms may be either symbols (i.e., identifiers) or
atomic-sorted function applications such as \texttt{(pubk~a)}.  Even
though an atom as a term may have terms within it, a receiver of an
atom is not allowed to extract terms that occur in it.  This reflects
the fact that the reception of the atom \texttt{(invk~k)}, the inverse
of some asymmetric key~\texttt{k}, does not allow the receiver to
construct~\texttt{k}.

Non-atomic terms are constructed by applications of encryption
(\texttt{enc}) and pairing (\texttt{cat}), where $n$-ary concatenation
is parsed right-associatively.  The second argument of an encryption
is the key. Encryption may also be written in an $n$-ary form where
the last argument is the key and the arguments preceding it are
implicitly concatenated.

A term carries one of its subterms if the possession of the right set
of keys allows the extraction of the subterm.  The carries relation is
the least relation such that (1)~$t$ carries~$t$, (2)~\texttt{(enc
  $t_0$ $t_1$)} carries~$t$ if~$t_0$ carries~$t$, and (3)~\texttt{(cat
  $t_0$ $t_1$)} carries~$t$ if~$t_0$ or~$t_1$ carries~$t$.  Note that
\texttt{(enc $t_0$ $t_1$)} does not carry~$t_1$ unless (anomalously)~$t_0$
carries~$t_1$.

\section{Protocols}

A protocol is a set of roles.
\begin{quote}
\begin{alltt}
(defprotocol \nterm{sym} basic \ntermp{role})
\end{alltt}
\end{quote}
The symbol \nterm{sym} names the protocol.  The symbol \texttt{basic}
identifies the term algebra used to specify messages in roles.

A role has the form:
\begin{quote}
\begin{alltt}
(defrole \nterm{sym} (vars \nterms{decl})
\quad (trace \ntermp{event})
\quad \ntermo{(non-orig \nterms{non})}
\quad \ntermo{(uniq-orig \nterms{atom})}
\quad \nterm{annos})

\nterm{non} ::= \nterm{atom} | (\nterm{height} \nterm{atom})
\end{alltt}
\end{quote}
Non-terminal \nterm{sym} is an S-expression symbol that names the
role.  A \nterm{decl} is a list of variable symbols followed by a sort
symbol.  The \texttt{trace} is a sequence of message events, each
indicating a message to be transmitted or received.  The syntax used
for a message event \nterm{event} has
one of two forms, \texttt{(send \nterm{term})} or \texttt{(recv
  \nterm{term})}.  The length of a role is the length of its trace,
and must be positive.  The remaining components of a role will be
described later.

A term originates in a trace if it is carried in some event
and the first event in which it is carried is a sending term.
A term is acquired by a trace if it first occurs in a receiving term
and is also carried by that term.

\section{Executions}

An execution of a protocol may involve any number of strands, each
conveying either regular or adversarial behavior.  Thus, each strand
is an instance of some role.  For CPSA input and output, a strand is
specified by the following form:
\begin{quote}
\begin{alltt}
(defstrand \nterm{sym} \nterm{int} \nterms{maplet})
\end{alltt}
\end{quote}
The symbol names the role, \nterm{int} is the height which must be
positive and no greater than the role's length, and the remainder
determines a substitution from role variables to terms.
\begin{quote}
\begin{alltt}
\nterm{maplet} ::= (\nterm{sym} \nterm{term})
\end{alltt}
\end{quote}
The trace associated with the specified behavior is the result of
truncating the role's trace so it agrees with the height, and applying
the substitution \texttt{(\nterms{maplet})}.

A strand's behavior includes inherited origination assumptions.  When
a role assumes atom~$a$ is uniquely originating using the
\texttt{uniq-orig} form, applying the substitution
\texttt{(\nterms{maplet})} to $a$ produces an inherited uniquely
originating atom.  Role atoms assumed to be non-originating using the
\texttt{non-orig} form are inherited similarly.  For a non-originating
assumption, a strand height may be associated with an atom.  In this
case, a non-originating assumption is inherited by strands that meet
or exceed the height constraint.  Note that the definition of a
uniquely originating atom and a non-originating atom in an execution
is still to come.

A strand in an execution is identified by a natural number.  To
describe an execution, the behavior of each participant is listed
sequentially, and position of the \texttt{defstrand} form in the list
determines the strand's identifier.  Zero-based indexing is used,
so zero identifies the first strand.

A messaging event in an execution occurs at a node, which is a pair of
natural numbers.  The first number is the strand's identifier.  The
second number is the position of an event in the trace of the
strand, once again using zero-based indexing.  Thus node \texttt{(1~1)}
in
\begin{quote}
\begin{alltt}
(defstrand r1 3 (a b) (b a))
(defstrand r2 2 (x a) (y a) (z b))
\end{alltt}
\end{quote}
names the last event in the last strand.  The term is the
result of instantiating the second event in role \texttt{r2}'s
trace using the substitution \texttt{((x a) (y a) (z b))}.

Message exchanges are part of an execution.  Each exchange is
described by a pair of nodes.  The first node must name a sending
term, and the second node must name a receiving term.  In an
execution, the two terms are the same.  Furthermore, for each
receiving term in a strand's trace, there is a unique node that
transmits its term.  In other words, all message receptions are
explained by transmissions within the execution.

In an execution, a \emph{uniquely originating atom} originates in the
trace of exactly one strand.  An inherited uniquely originating atom
must originate in the trace of its strand.  CPSA uses uniquely
originating atoms to model freshly generated nonces used in many
protocols.

A \emph{non-originating atom} is carried by no trace of any strand in
an execution, and it or its inverse is the key of an encryption in one
of those traces.  The inherited non-origination atoms must satisfy
this property too.

Strands in executions represent both adversarial and non-adversarial
behaviors.  A strand that is an instance of a protocol role is
non-adversarial, and is called regular.  A strand that represents
adversarial behavior is called a penetrator strand.

The roles that define adversary behavior codify the basic abilities
that make up the Dolev-Yao model.  They include transmitting an atom
such as a name or a key; transmitting a tag; transmitting an encrypted
message after receiving its plain text and the key; and transmitting a
plain text after receiving ciphertext and its decryption key.  The
adversary can also concatenate two messages, or separate the pieces of
a concatenated message.  Since a penetrator strand that encrypts or
decrypts must receive the key as one of its inputs, keys used by the
adversary---compromised keys---have always been transmitted by some
strand. The basic adversary roles are built into CPSA.

\section{Skeletons}

CPSA never directly represents adversarial behavior.  Instead, a
skeleton is used.  A skeleton represents regular behavior that might
make up part of an execution. A skeleton is specified in CPSA output using a
\texttt{defskeleton} form.
\begin{quote}
\begin{alltt}
(defskeleton \nterm{sym} (vars \nterms{decl})
\quad \ntermp{defstrand}
\quad \ntermo{(precedes \nterms{pair})}
\quad \ntermo{(non-orig \nterms{atom})}
\quad \ntermo{(uniq-orig \nterms{atom})})
\end{alltt}
\end{quote}
The symbol names the protocol used by its participants.  The regular
strands are specified as they are in an execution.  The precedes form
defines a binary relation on nodes (\texttt{\nterm{pair} ::=
  (\nterm{node} \nterm{node})}).  As in an execution, the first node
names a sending term and the second term names a receiving term.
Unlike an execution, the pair of nodes in the relation need not agree
on their message term.  Two nodes are related if the sending event
precedes the reception reception event, as an execution it represents
may include events in between.

The final two additional components of a skeleton are a set of
non-originating atoms, and a set of uniquely originating atoms.  To be
a skeleton, each uniquely originating atom must originate in at
most one strand in the skeleton, and each non-originating atom must
never be carried by some event in the skeleton and every
variable that occurs in the atom must occur in some event.
Furthermore, for each uniquely originating atom that originates in the
skeleton, the node relation must ensure that reception nodes that
carry the atom follow the node of its origination.

One special skeleton is associated with each execution.  It summarizes
the regular behavior of the execution.  It is derived from the
execution by enriching its node relation to contain all node orderings
implied by transitive closure, deleting all strands and nodes that
refer to penetrator behavior, and then performing the transitive
reduction on the resulting node relation.  The set of uniquely
originating atoms is the set of terms that originate on exactly one
strand in the execution, and are carried in a term of a regular
strand.  The set of non-originating atoms is the union of two sets.
One set contains each term that is used as an encryption or decryption
key in some term in the execution, but is not carried by any term.
The other set contains the terms specified by non-origination
assumptions in roles.  If a realized skeleton instance maps all of the
variables that occur in one of its non-originating role terms, the
mapped term is a member of the skeleton's set of non-originating
terms.  A skeleton is \emph{realized} if it summarizes the behavior of
some execution.

\subsection{Preskeletons}

Preskeletons are used to pose problems for CPSA to solve.  A
preskeleton is similar to a skeleton except atoms assumed to uniquely
originate may originate in more than one strand, and the node relation
need not ensure that reception nodes that carry the atom follow some
node of origination.  Experience has shown that it is more natural to
specify some problems in a form that doesn't satisfy all the
properties of a skeleton.  If CPSA cannot immediately convert its
input into a skeleton, an error is signal.  With the exception of the
restatement of the original problem, all preskeletons printed by CPSA
are skeletons.  A problem statement is called a \emph{scenario}, and
the converted skeleton is called the \emph{scenario skeleton}.

\subsection{Shapes}

Given a scenario skeleton, CPSA determines whether there is an execution
containing the strands in the skeleton, and satisfying its
origination assumptions.  Usually an execution contains additional
regular strands, as well as adversary behavior.  A major part of what
CPSA does is to find all additional regular strands that are necessary
to extend the scenario to an execution---a realized skeleton.  If a
realized skeleton is most-general, in the sense that there is no other
realized skeleton that can be instantiated to it by merging nodes or
atoms, then it is called a {\em shape}. CPSA finds all shapes for a
scenario.

\section{Listeners}

In addition to the roles specified in a protocol, for each term~$t$, a
regular strand may be an instance of a so-called {\em listener} role
with the trace \texttt{(recv~$t$) (send~$t$)}.  There are no
non-originating or uniquely originating atoms associated with a
listener role.  Listener behavior is specified with:
\begin{quote}
\begin{alltt}
(deflistener \nterm{term})
\end{alltt}
\end{quote}

A listener strand is used in a skeleton to assert that a term~$t$ is
derivable by the adversary, unprotected by encryption.  Hence it is used
to test for compromise of a term.  The term is protected if the
resulting skeleton is unrealizable.  Otherwise, CPSA will find a shape
that shows how the adversary accesses~$t$.

\section{Annotations}

To be analyzed, each role in a protocol must include an
\texttt{annotations} form, as defined in Table~\ref{tab:anno}.  The
\nterm{term} just after the \texttt{annotations} symbol is a role atom
that, when instantiated, is the principal associated with the strand in
the shape.  A principal may be a key.

\begin{table}
\begingroup\ttfamily
\begin{tabular}{rcl}
\nterm{annos}&$::=$
&(annotations \nterm{term} (\nterm{int} \nterm{form})$^\ast$)
\\ \nterm{form}&$::=$&(\nterm{sym} \nterms{fterm}) | (not \nterm{form})
\\ &|& (and \nterms{form}) | (or \nterms{form})
\\ &|& (implies \nterms{form} \nterm{form})
\\ &|& (iff \nterm{form} \nterm{form})
\\ &|& (says \nterm{term} \nterm{form})
\\ &|& (forall (\nterms{decl}) \nterm{form})
\\ &|& (exists (\nterms{decl}) \nterm{form})
\\ \nterm{fterm}&$::=$&\nterm{term} | (\nterm{sym} \nterms{fterm})
\end{tabular}
\endgroup
\caption{Annotation Syntax}\label{tab:anno}
\end{table}

What follows is sequences of pairs.  The integer gives the position of
the event in the trace that is annotated by the formula, using
zero-based indexing.  Thus, each formula is associated with a node.
Nodes for which no formula is specified are implicitly defined to be
the trivial formula \texttt{(and)} for truth.  Use \texttt{(or)} for
falsehood.

The language of formulas is first-order logic extended with a modal
``says'' operator. Formula terms may include function symbols that are not
part of a protocol's message signature.

On output, each shape contains an \texttt{annotations} form and an
\texttt{obligations} form. The annotations form presents every
non-trivial formula derived from the protocol. The obligations form
presents every non-trivial formula that must be true if the shape is
sound.

\end{document}
